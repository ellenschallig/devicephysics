\section{Summary: relevance of the devices}

Given the title of the article, ``A bright future for organic field-effect transistors", it is clear that the devices in general sense are deemed very important. Yet there is still a lot of work to be done, so that the article has to differentiate in different device generations and design concepts. There is an important role for synthetic chemistry in tailoring the processing conditions and the functional properties of materials. In many cases there is a trade-off between high performance and ease of fabrication.

The general objective is to allow the fabrication of multifunctional organic devices using extremely simple device structures. This has strong economical relevance. \textsc{olet}s represent a step forwards.

As compared to \textsc{oled}s, a clear advantage is their potentially higher electroluminescence quantum efficiency. Moreover they can be used to investigate fundamental opto-electronic properties in organic materials, and can be used in all kinds of applications, ranging from display technology to electrically driven organic lasers. In these applications they may enable the development of next-generation technologies, such as organic active matrix display technology.

As \textsc{olet}s are fully compatible with many well-established electronic and photonic technologies, they can have a huge impact on society, especially because they can be fabricated on plastic substrates in combination with solution processing. This enables relatively low-cost production, which makes the economic driving forces huge.