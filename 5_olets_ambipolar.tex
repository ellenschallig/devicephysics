\subsection{Ambipolar OLETs} %Cees|Ellen
In unipolar devices only one type of charge carrier is transported efficiently across the transistor channel and light generation is restricted to an area close to the minority carrier injection electrode. Ambipolar organic semiconductor devices don not have this limitation. This means that they can be fabricated of complementary logic circuits with a single active layer. Ambipolar charge transport is crucial in LETs for maximizing exciton recombination through electron-hole balance as well as for adjusting the position of the recombination region in the channel by tuning the gate voltage. The consequent decrease of exciton quenching leads to improved quantum efficiency of the emission from the device.

In principle, pure organic semiconductors should support both electron and hole conduction equally. However in practice most of the organic semiconductor films only display unipolar charge transport. Of these the majority is p-type, so holes are more mobile than electrons). Ambipolar field-effect charge mobility of electroluminescent organic thin films seems to be limited and their optoelectronic response in a device structure remains to be explored.

For instance, as the field-effect mobility determines the switching time of the OLET device and is a critical parameter for all those applications where light emission is to be modulated by the applied voltage, e.g. for active matrix displays and frequency-modulated nano-scale light sources.

In view of the limited number of electroluminescent materials with good ambipolar mobility values, alternative approaches to achieve high ambipolar transport in OLETs have been explored. The first ambipolar OLETs were demonstrated using a bulk heterojunction as the active component of the device. Like previously applied in LEDs, solar cells and FETs, this uses the mixing of two materials with complementary properties.

The most important requirement that the two materials must comply with, is that the relative positions of the HOMO and LUMO must allow exciton formation and recombination in the material with the smaller energy gap. Given this restriction, the approach of the bulk heterojunction can be extended to other materials in the search for higher-brightness OLET devices). The electroluminescense intensity is also determined by the relative concentration of materials and by the structural features of the bulk heterojunction. However, the absolute mobility values are low.

OLETs based on two component layered structures have been realized. Here growth compatibility between the n-type and p-type materials is essential in forming a continuous interface and in controlling the resulting optoelectronic response of the OLET. Therefore the sequence of the deposition of the layers is important for determining the device characteristics. The optimum performance is not necessarily achieved by using materials with the highest mobility values in single-layer devices. 

To date, the bilayer approach provides the highest balanced mobility values in ambipolar OLET devices. However the separation between the electron transport layer and the hole transport layer, in spite of the electron-hole attraction, drastically reduces the probability of electrons and holes meeting to form excitons inside the device channel. When charges are accumulated at the interface between the dielectric substrate and the organic material, near the drain top electrode, the superposition of gate contact/bilayer/drain contact is likely to form an LED structure at the origin of the light emission, which is therefore confined next to the drain metal electrode as in the case of unipolar OLETs.

Recently it was shown that in the polymer-based ambipolar LETs the emission zone can be scanned across the transistor channel on changing the gate bias. This was done with low-workfunction metals for the electron-injection electrode and of a high-workfunction metal for the hole-injection electrode. In addition, trap-free dielectrics were used to avoid electron trapping and enable ambipolar transport. 

The observation of a spatially resolved recombination zone whose location in the transistor channel is controlled by the applied bias, demonstrates the coexistence of electron and hole channels, and therefore the ambipolar nature of charge transport. The recombination zone is located at the centre of the channel when electron and hole currents are balanced. The electroluminescense quantum efficiency can be as high as that of LEDs based on the same material. 

At present, the drawback of the ambipolar polymer-based LETs remains the mobility values. However, the latest results, together with the continuous development of the knowledge of the chemical/physical properties of the relevant interfaces in the device, and the possibility of chemically tailoring the active materials, open exciting perspectives for the full realization of the scientific and technological potential of OLETS.




