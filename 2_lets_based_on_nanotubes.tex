\subsection{LETS based on Carbon nanotubes} %Armin|Cees

To prepare novel nanoscale light sources for use in fully integrated optoelectronic circuits, there are several targeted methods. One of them is engineering of light-emitting nanowires made of direct-bandgap semiconductors. [15] have made great advancements in this field by assembling p-doped and n-doped nanowires to form a p-n junction. [16] achieved this by fabricating nanowire superlattices. Unfortunately for achieving high performance levels, also with these methods fabrication ease is lost. 

An alternative approach is to use semiconducting single walled carbon nanotubes as the as the active component in a FET [17]. Based on these components, carbon nanotube FETs of n-type, p-type and ambipolar character, as well as logic gates based on these components have been fabricated [18,19,20].

Under certain conditions a LET emitting infrared light can be constructed as shown in figure xx. The carbon nanotube FETs exhibit ambipolar charge transport, which follows simultaneous injection of holes and electrons via thermally assisted tunnelling through the Schottky barriers formed at the source and drain contacts. Infrared light emission is possible under correct bias conditions and ambipolar transport with balanced electron and hole currents.

The infrared radiation emitted bij the carbon nanotube LET has several properties. Due to the elongated shape of the tube the light is polarised parallel to the tubes axis. The bandgap of the nanotube is inversely proportional to its diameter, resulting in different wavelength radiation. For example a diameter of 1.4nm, 1nm and 0.8nm will result in a wavelength of 1650nm 1200nm and 1000nm respectivilty. Some tenability is expected for the future [21] by changing diameter sizes, yet the range of possibilities has yet to be explored.
