\subsection{LETS based on Carbon nanotubes} %Armin|Cees
To prepare novel nanoscale light sources for use in fully integrated optoelectronic and photonic circuits there are several targeted methods. One of them is engineering of light-emitting nanowires made of direct-bandgap semiconductors. [15] have made great advantages in this field by assembling p-doped and n-doped nanowired to form a p-n junction. [16] achieved this by fabrication nanowire superlattices. Though achieving high performance the downside of these methods is the difficulty to fabricate them. [18]-[20] used a different approach. They were able to produce carbon nanotube FETs. In these FETs a semiconducting single walled carbon nanotube is used as active component. In this way it is possible to produce different types of FETs. 
Under certain conditions a LET emitting infrared light can be constructed as shown in figure xx. The carbon nanotube FETs exhibit ambipolar charge transport, which follows simultanious injection of holes and electrons via thermally assisted tunnelling through the Schottky barriers formed at the source and drain contacts. Infrared light emission is possible under correct bias conditions and ambipolar transport with balanced electron and hole currents.

The infrared radiation emitted bij the carbon nanotube LET has several properties. Due to the elongated shape of the tube the light is polarised parallel to the tubes axis. The bandgap of the nanotube is inversely proportional to its diameter, resulting in different wavelength radiation. For example a diameter of 1.4nm, 1nm and 0.8nm will result in a wavelength of 1650nm 1200nm and 1000nm respectivilty. Some tenability is expected for the future [21] by changing diameter sizes, yet the range of possibilities has yet to be explored.
