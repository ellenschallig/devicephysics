\subsection{Unipolar OLETs}\label{sec:unipolar_olets} %Cees|Ellen

\textsc{olet}s were first demonstrated by Hepp et al.~using a tetracene thin film. Tetracene is the four-ringed member of the series of acenes, a class of organic compounds and polycyclic aromatic hydrocarbons made up of linearly fused benzene rings. Moreover, it is a molecular organic semiconductor.

Typical current carriers in organic semiconductors are holes and electrons in $\pi$-bonds. Almost all organic solids are insulators. But when their constituent molecules have $\pi$-conjugate systems, electrons can move via $\pi$-electron cloud overlaps, especially by hopping, tunnelling and related mechanisms. Also polycyclic aromatic hydrocarbons work with this mechanism.

To make the first \textsc{olet}, Hepp used interdigitated gold source and drain electrodes, fabricated on a Si/SiO$_2$ substrate before deposition of the organic active layer. The charge transport and light emitting layer is the tetracene, configured as a polycrystalline film. The resulting electrical characteristics are typical of a $p$-type \textsc{fet}. Light emission from the tetracene indicates that electrons and holes are simultaneously injected into the active layer. The electrons cannot move through the tetracene, so they are trapped at the gold/tetracene interface. At this interface they then recombine with the holes, emitting light.

One important issue that can be relevant for other materials under the same driving scheme, is that electrons are injected from gold into tetracene over an theoretically unexpected nominal barrier of 2.7 eV. This points to the actual structure of the gold/tetracene interface, where a composite layer could be formed. Thus it is no simple matter of considering the different materials' energy levels separately. 

A number of investigations were done to optimize this first device and to make it ambipolar. Yet never electron transport was observed in the tetracene thin films. The primary limiting process for achieving high-brightness emission is singlet-triplet quenching. This was concluded after numerical simulations looking at exciton processes. Triplets appear to be most dominant in quenching singlets. It prevents pure tetracene films, when provided with a realistic optical feedback structure, to reach the threshold for stimulated emission.

\textsc{olet}s' electroluminescent and electrical properties are, among other things, dependent on the transistor channel length. On decreasing the channel length, both electron injection and electroluminescence efficiency are improved. This unfortunately holds up only to a certain point. The contact resistance from a typical metal/organic interface tends to dominate the electrical characteristics of the transistor, hindering further improvement. Unfortunately the external quantum efficiency is still very low, so that there are no practical applications. 

Organic materials with both efficient electroluminescence and transistor characteristics are needed. The application of well-established \textsc{oled} materials is not straightforward, as most of them have low \textsc{fet}-performance. Their strong molecular packing that allows high mobility, increases non-radiative decay.

As an alternative spin-coated polymers have been used as active layers in \textsc{let}s with bottom-contact device structures as in figure \ref{fig:LET}. The polymers used are among the most commonly used \textsc{led} polymers. In addition to extending \textsc{olet} concepts to polymers, experiments show a clear increase in light-emission efficiency on using metals for source and drain electrodes that have a workfunction more suitable for holes and electrons injections respectively.

As mentioned before, to fabricate large-area and low-cost devices, the good solubility of the organic semiconductor, which would allow printing and casting processes, is important. A new molecular system was devised which is suitable for drop-casting onto a pre-patterned \textsc{fet} structure to produce \textsc{olet} devices. Although the transistor characteristics are less than if produced by vacuum sublimation, most likely due to morphological characteristics of the solution-processed films and to the lower degree of molecular ordering, it widens the range of processing techniques and again points to the crucial role played by synthetic chemistry in tailoring the processing conditions and functional properties of materials.