\section{The article (10 pages)}
\textbf{Describe the main message(s) of the paper. In case of longer papers, select only those parts which you think are relevant in relation with the course material. A detailed overview of all issues discussed in the paper is not required.}

Since the separate discoveries of the Field-effect transistor in the early 20th century, by Lilienfeld in 1926 and then by Oskar Heil in 1934, no practical applications arose before the transistor effect was observed and explained by Shockley in 1947. Currently the rate of new devices is much higher.

Recent advancements in organic science induced the development of a broad range of devices, such as organic light-emitting diodes (OLEDs) solar cells and organic field-effect transistors (OLETs). This has economic viability since they can be produced as low-cast, large-area, lightweight and as more flexible and hence more integrated devices. The latter is important because steps required to construct highly integrated opto-electronic systems of separate devices affect the simplicity of the system architecture and production cost. 

Field-effect transistors are emerging as useful devices for efficient light generation from inorganic semiconductors, carbon nanotubes and organic thin films. In this progress article, the focus lies on organic light-emitting field-effect transistors and the role played by the material properties, device features and the active layer structure in determining device performance.

