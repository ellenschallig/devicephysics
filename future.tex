\section{Future}\label{sec:future}
The article mostly focuses on the discoveries made in the \textsc{olet}s field and future possibilities. It shows a bright future for \textsc{olet}s, mainly praised for their low-cost and very favourable production methods. The possibility to make flexible devices creates a whole new range of possible applications.

However most of the achievements are obtained in laboratory environments, to take advantage of the low-cost possibilities it is necessary to produce \textsc{olet}s on an industrial scale, which was still not the case in 2006.

Nowadays organic photonic devices are mass-produced. Only these are not \textsc{olet}s but \textsc{oled}s. Many mobile phones that are produced contain an \textsc{oled} display that match the performance of inorganic displays while the price is in the same range. Full size \textsc{oled} television screens are also available on the consumer market, but the price is almost ten times higher than that of inorganic (\textsc{lcd}) displays. The expectation is that the prices of big \textsc{oled} displays will drop further. 

Also in the field of \textsc{olet}s, a lot of progress has been made since 2006. \citet{Capelli} published a paper on \textsc{olet} performance. They report improved \textsc{olet} devices that are more than 100 times more efficient than equivalent \textsc{oled}s and more than 2 times more efficient than the optimized \textsc{oled} with the same emitting layer. The devices are 10 times more efficient than any other reported \textsc{olet}s. The device has a very high quantum efficiency of 5\% compared to \textsc{oled}s (2.2\%) and previous \textsc{olet} designs (0.6\%). 

Nevertheless, despite the article's title, one of the remaining weaknesses that needs improvement is the brightness.
