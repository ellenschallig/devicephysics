\section{Future}\label{sec:future}
The article mostly focuses on the discoveries made in the OLETs field and future possibilities. It shows a bright futre for OLETs. Mainly praised for their low-cost and very favorable production methods. The possibility to make flexibla devices creates a whole new range of possible applications. However most of the achievments are obtained in laberatorium environments, to take advantage of the low-cost possibilities it is necessary to produce OLETs on an industrial scale, which was still not the case in 2006. 

However nowadays organic photonic devices are massproduced. Only these are not OLETs but OLEDs. Many mobile phones that are produced contain an OLED display that match the performance of inorganic displays while the price is in the same range. Fullsize OLED television screens are also available on the consumer market, but the price is almost ten times higher than that of inorganic (LCD) displays. The expectation is that the prices of big OLED displays will drop further. In the field of OLETs there is also a lot of progress made since 2006. 

\citet{Capelli} published a paper on the performance of OLETs. They report improved OLET devices that are more than 100 times more efficient than equivalent OLEDs and more than 2 times more efficient than the optimized OLED with the same emitting layer. The devices are 10 times more efficient than any other reported OLETs. The device has a very high quantum efficiency of 5\% compared to OLEDs (2.2\%) and previous OLET designs (0.6\%). One of the weaknesses that still has te be improved is the brightness.
