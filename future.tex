\section{Future}\label{sec:future}
The article mostly focuses on the discoveries made in the OLETs field and future possibilities. It shows a bright future for OLETs, mainly praised for their low-cost and very favourable production methods. The possibility to make flexible devices creates a whole new range of possible applications. However most of the achievements are obtained in laboratory environments, to take advantage of the low-cost possibilities it is necessary to produce OLETs on an industrial scale, which was still not the case in 2006.

However nowadays organic photonic devices are mass-produced. Only these are not OLETs but OLEDs. Many mobile phones that are produced contain an OLED display that match the performance of inorganic displays while the price is in the same range. Full size OLED television screens are also available on the consumer market, but the price is almost ten times higher than that of inorganic (LCD) displays. The expectation is that the prices of big OLED displays will drop further. 

Also in the field of OLETs, there a lot of progress has been made since 2006. \citet{Capelli} published a paper on OLET performance. They report improved OLET devices that are more than a 100 times more efficient than equivalent OLEDs and more than 2 times more efficient than the optimized OLED with the same emitting layer. The devices are 10 times more efficient than any other reported OLETs. The device has a very high quantum efficiency of 5\% compared to OLEDs (2.2\%) and previous OLET designs (0.6\%). Nevertheless, despite the article's title, one of the remaining weaknesses that need improvement is the brightness.
