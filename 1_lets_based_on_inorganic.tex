\section{The Paper NEEDS DIFFERENT TITLE}

\subsection{LETs based on inorganic semiconductors} %Ellen|Armin


\citep{Muccini}

Silicon is the most well known and most successful material in the field of electronics. However, its current use in opto-electronics is limited because of its poor light-emitting properties. This is because silicon is an indirect-bandgap semiconductor, which makes it unfavourable for light emission. For each radiative electron-hole recombination, a phonon is needed to conserve momentum. This causes dominance of non-radiative recombination mainly occurring at lattice defect sites, such as dislocations, impurities, cluster defects and precipitates.

This problem is overcome by using silicon nanocrystals instead of bulk silicon. For nanocrystals hole and electron diffusion is limited by the size of the crystals. This way defect sites become isolated and radiative emission is the dominant recombination pathway. Emission from nanocrystals can therefore be rather high, up to a quantum efficiency of  10\% [ref10]. Additionally, the emitted light can be tuned by the exact size of the nanocrystals, from near-infrared to the visible range. 

To achieve such electrically driven bright emission, an efficient way to inject opposite charges into the nanocrystals is required. The most successful approach has been a FET structure in which nanocrystals are embedded into the gate oxide [ref11]. Electrons and holes are injected by tunnelling through the oxide under application of an alternating electric field. This induces sequential accumulation of electrons and holes in the nanocrystals, and light is generated at each cycle.

In this FET structure the electric field required to generate light is lower than in the LED 'sandwich' structure. This shows why silicon still has great potential to be explored.

However, direct-bandgap inorganic semiconductors are better for applications where opto-electronic performance faces more severe constraints. Materials such as gallium arsenide (GaAs) and indium phosphide (InP) can generate much brighter luminescence and allow higher modulation speeds than silicon. LETs based on an InGaP/GaAs heterojunction have been demonstrated [ref12], showing that LETs based on inorganic semiconductors could be used for display and communication purposes. In a further development the InGaP/GaAs LET was designed as a transistor laser.

Although the performances of inorganic semiconductor LETs and transistor lasers are the best achieved to date, their fabrication process is rather complicated, compatibility with other technology is limited and they are relatively expensive. Hence, it is useful to look at different devices.
