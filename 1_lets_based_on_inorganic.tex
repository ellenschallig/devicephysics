\subsection{LETs based on inorganic semiconductors} %Ellen|Armin

Silicon is the best known and dominant material in the field of electronics. However, it has limitations because of its poor light-emitting properties. Bulk silicon is an indirect-bandgap semiconductor, which makes it unfavourable for light emmission as there are phonos needed for radiative electron-hole recombinations. Because of this requirement, highly mobile charge carriers preferentially occur at defect sites via non-radiative processes, therefore not emitting any light.

In silicon nanocrystals the diffusion is limited by the size of the crystals, and radiative emission is the dominant method of recombination. The quantum efficiency for emission in these crystals approaches 10\%. [ref10] The emitted light can be tuned by the exact size of the nanocrystals, from near-infrared across the visible range. The most succesful approach has been a field-effect transistor structure in which nanocrystals are embedded into the gate oxide. [ref11] (Maybe not include this?: Electrons and holes are injected by tunneling, through the application of an alternating electric field. This induces alternating accumulation of electrons and holes in the nanocrystals, which recombine every cycle and produce light.)

So silicon still has a great potential to be explored.

However, inorganic direct-bandgap semiconductors are better for applications with tighter constraints; materials such as gallium arsenide (GaAs) or indium phosphide (InP) can generate much brighter luminescence and allow higher modulation speed than silicon. Light-emitting transistors (LETs \textbf{al in de nog te schrijven inleiding de afkortingen noemen}) based on a InGaP/GaAs heterojunction have been demonstrated. [ref12] This result demonstrated that LETs based on inorganic semiconductors (are three-terminal frequency-modulated light sources that) could be used for display and communication purposes. In a further development the InGaP/GaAs LET was designed as a transistor laser, with much lower modulation speeds than state-of-the-art heterojunction bipolar transistors. [ref13,14]

Although the performances of inorganic semiconductor LETs and transistor lasers are the best achieved to date, the fabrication process is very complicated and relativly expensive. Therefore it is useful to look at other approaches.
