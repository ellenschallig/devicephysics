\subsection{LETs based on inorganic semiconductors} %Ellen|Armin

Silicon is the most well known and most successful material in the field of electronics. However, its current use in opto-electronics is limited because of its poor light-emitting properties. This is because bulk silicon is an indirect-bandgap semiconductor, which makes it unfavourable for light emission. For each radiative electron-hole recombination, a phonon is needed to conserve momentum. This causes dominance of non-radiative recombination mainly occurring at lattice defect sites, such as dislocations, impurities, cluster defects and precipitates.

This problem is overcome by using silicon nanocrystals instead of bulk silicon. For nanocrystals hole and electron diffusion is limited by the size of the crystals. This way defect sites become isolated and radiative emission is the dominant recombination pathway. Emission from nanocrystals can therefore be rather high, up to a quantum efficiency of  10\% [ref10]. Additionally, the emitted light can be tuned by the exact size of the nanocrystals, from near-infrared to the visible range. 

To achieve such electrically driven bright emission, an efficient way to inject opposite charges into the nanocrystals is required. The most successful approach has been a FET structure in which nanocrystals are embedded into the gate oxide [ref11]. Electrons and holes are injected by tunnelling, through the application of an alternating electric field. This induces alternating accumulation of electrons and holes in the nanocrystals, which radiatively recombine every cycle.

So silicon still has a great potential to be explored.

However, inorganic direct-bandgap semiconductors are better for applications with tighter constraints; materials such as gallium arsenide (GaAs) or indium phosphide (InP) can generate much brighter luminescence and allow higher modulation speed than silicon. Light-emitting transistors (LETs \textbf{al in de nog te schrijven inleiding de afkortingen noemen}) based on a InGaP/GaAs heterojunction have been demonstrated. [ref12] This result demonstrated that LETs based on inorganic semiconductors (are three-terminal frequency-modulated light sources that) could be used for display and communication purposes. In a further development the InGaP/GaAs LET was designed as a transistor laser, with much lower modulation speeds than state-of-the-art heterojunction bipolar transistors. [ref13,14]

Although the performances of inorganic semiconductor LETs and transistor lasers are the best achieved to date, the fabrication process is very complicated and relativly expensive. Therefore it is useful to look at other approaches.
