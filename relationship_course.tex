\section{Relationship to the course (3 to 4 pages)}
Summarize the themes/aspects from the lecture course which are relevant for the paper which you will discuss. 



\subsection{Van Wees}
From the slides, reading material. Van Wees first because he was most fundamental about devices? Building up towards the more advanced organic stuff from Loi (as in the article).



\subsection{Loi}
The Organic and optoelectronic part of the lecture is of most relevance to the paper. After the introduction into (inorganic) Light Emmitting Transistors the paper focuses on organic devices. 

\subsubsection{Tunability}
One of the aspects from the lectures that is also covered in the paper is the tunabilty of organic semiconductors. With bulk inorganic semiconductor material it is not possible to change the bandgap a lot. Especially in light producing devices it is necessary to make nanocrystals. These nanocrystal structures are also needed for indirect bandgap meterials like silicon because they need a phonon to complete charge recombination. Without the nanocrystal structure the rocombination will occur via an non-radiative process. For organic semiconductors however it is relativly easy to change the bandgap. Most of the organic materials have a relative complex molecular structure compared to inorganic materials, therefore there are more possibilities to adjust the molecule slightly rusulting in different properties. 
\subsubsection{Easy and low-cost production}
\subsubsection{Other differences between organic and inorganic devices}
-Mobility of charge
-Lower efficiency for organic devices
From the slides, reading material.



\subsection{Tamalika}
Perhaps some historical perspective?
