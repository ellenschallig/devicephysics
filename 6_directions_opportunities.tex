\subsection{Directions and opportunities} %Ellen|Armin

OLETs are of high interest as they can be used in many different applications, from communication systems to solid-state lighting, and provide a novel device architecture to investigate fundamental optoelectronic properties in organic materials. The in-plane architecture of OLETs allows for direct probing and observation of various fundamental aspects of organic material science, as opposed to the OLED technology.

OLETs may be the key element for the next-generation organic active matrix display technology. They can provide effective solutions for the brightness and lifetime of the electroluminescent pixels, through the high degree of control of charge injection and accumulation in the organic layer. The integration of light-emitting and electrical switching in one element reduces the amount of other elements to be produced, which results in a cheaper fabrication of the active matrix display.

The in-plane OLET structure can be a valid alternative for the vertical OLED structure for an electrically pumped organic laser. These lasers have many advantages over the III-V semiconductor lasers, such as lower cost and wider range of possible lasing emission. However exciton quenching and photon losses are still major problems, which have prevented the realization of the electrically pumped organic laser to date.

The LET configuration can prevent exciton quenching and photon losses by moving the metal electrodes away from the region of exciton formation and light emission. The high mobility of carriers in a FET configuration minimizes the required charge-carrier densities, which also(?) reduces exciton quenching and photon absorption. These key advantages make the field-effect transistor approach the most favourable for achieving the organic laser.

OLETs have potentially higher light-emission efficiency and brightness than OLEDs, which opens the market of low power consumption solid state lighting. OLETs also can be fully compatible with well-established other technologies which may allow for the development of optical communication with OLETs as a key active element. Finally, OLETs are soluble(?) and can be fabricated on plastic surfaces [ref57]. This is a gateway for the printing on large-area and flexible substrates.
