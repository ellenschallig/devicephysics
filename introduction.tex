\section{Introduction}
Field-effect transistors are emerging as useful devices for efficient light generation from a variety of materials. In this progress article from \citet{Muccini}, the focus lies on organic light-emitting field-effect transistors and the role played by the material properties, device features and the active layer structure in determining device performance.

Since the separate patenting of the field-effect transistor (\textsc{fet}) in the early 20th century, by Lilienfeld in 1926 and then by Heil in 1934, no practical applications arose before the transistor effect was observed and explained by Shockley in 1947.

Recent advancements in organic science induced the development of a broad range of devices, such as organic light-emitting diodes (\textsc{oled}s), solar cells, and organic field-effect transistors (\textsc{ofet}s). This has economic viability since they can be produced as low-cost, large-area, lightweight and more flexible devices than their inorganic counterparts.

Organic light-emitting field-effect transistors (\textsc{olet}s) are an example of more integrated devices, combining the electrical switching functionality of a field-effect transistor and the capability of light generation into a single device. This is important because steps required to construct highly integrated opto-electronic systems of separate devices affect the simplicity of the system architecture and production cost.

Moreover \textsc{olet}s offer an ideal structure for improving the lifetime and efficiency of organic light emitting materials by different driving conditions than standard \textsc{oled}s (section \ref{sec:thinfilms}). Also they can achieve optimized charge-carrier balances (section \ref{sec:unipolar_olets}).

In the article several recent advances of \textsc{fet} architectures to achieve light generation from different organic materials are reviewed. First it presents an overview of comparable inorganic devices, such as those based on silicon and direct-bandgap semiconductors. Next the emission properties of single-walled carbon nanotube \textsc{fet}s are discussed. They have potential for use as highly integrable nanoscale light sources. \textsc{olet}s based on organic thin films are discussed for their low production cost and ease of integration on virtually any substrate. Finally possible directions of development are examined.

First however, a short recap is given on the basics. Transistors and $p$--$n$ junctions are discussed, and a comparison between organic and inorganic semiconductors is given.