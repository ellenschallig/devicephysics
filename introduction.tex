\section{Introduction}
Field-effect transistors are emerging as useful devices for efficient light generation from inorganic semiconductors, carbon nanotubes and organic thin films. In this progress article, the focus lies on organic light-emitting field-effect transistors and the role played by the material properties, device features and the active layer structure in determining device performance.

Since the separate discoveries of the field-effect transistor (FET) in the early 20th century, by Lilienfeld in 1926 and then by Oskar Heil in 1934, no practical applications arose before the transistor effect was observed and explained by Shockley in 1947. %Currently the rate of new devices is much higher.

A transistor is a semiconductor device that can amplify and switch electronic signals. It is the key component in practically all modern electronics. The field-effect transistor uses an electric field to control the electrical conductivity of a material and thus influence the switching and amplifying properties.

Recent advancements in organic science induced the development of a broad range of devices, such as organic light-emitting diodes (OLEDs) solar cells and organic field-effect transistors (OLETs). This has economic viability since they can be produced as low-cost, large-area, lightweight and more flexible devices which integrate functionalities currently only possible by combining different devices. The latter is important because steps required to construct highly integrated opto-electronic systems of separate devices affect the simplicity of the system architecture and production cost.

OLETs are an example of more integrated single devices, combining the electrical switching functionality of a field-effect transistor and the capability of light generation. They could pave the way to even more. 

Moreover OLETs offer an ideal structure for improving the lifetime and efficiency of organic light emitting materials by different driving conditions than standard OLEDs \textbf{explain!}. Also they can achieve optimized charge-carrier balances \textbf{explain!}.

In this progress article several recent advances of FET architectures to achieve light generation from different organic materials are reviewed. First it presents an overview of comparable devices, such as those based on silicon and direct-bandgap semiconductors. Next the emission properties of single-walled carbon nanotube FETs are discussed. They have potential for use as highly integrable nanoscale light sources. OLETs based on organic thin films are discussed for their low production cost and ease of integration on virtually any substrate. 
